\documentclass[twocolumn]{report}
\usepackage{graphicx}
\usepackage{tocloft}
\usepackage{placeins}
\usepackage{siunitx}
\usepackage{subfiles}
\graphicspath{ {./images/} }

\title{\textbf{Dynamic Audio System Based On Listener's Position For Surround Sound Effect}}
\author{Nachiket Kamod (12), N V Roshni (31),\\Ganeshsai Muppasani (30), Jagrati Chowdhari (9), \\\\\textbf{Guided By,}\\Prof. Deepali Yewale}
\date{\textbf{Academic Year}\\2020 - 21}

\begin{document}

\maketitle

\tableofcontents

\listoffigures

\chapter{}

\section{INTRODUCTION}

\subfile{subfiles/chapter_1/introduction/introduction.tex}

\section{ABSTRACT}

\subfile{subfiles/chapter_1/abstract/abstract.tex}

\chapter{LITERATURE SURVEY}

\subfile{subfiles/chapter_2/literature_survey/literature_survey.tex}

\section{CONCLUSION}

\subfile{subfiles/chapter_2/conclusion/conclusion.tex}

\chapter{AIM AND OBJECTIVES}

\section{Aim}

To develop a real-time self-adjusting Audio system based on listener's 
position to achieve a high-quality air sound effect.

\section{Objectives}

\begin{description}
    
    \item[1.]To introduce automation and artificial intelligence into the current 
    trend of audio systems.
    \item[2.]To make the audio system compatible with adjusting it’s orientation 
    and sound intensity based on the listener’s position.

\end{description}

\section{Methodology}

\subfile{subfiles/chapter_3/methodology/methodology.tex}

\section{Specifications of the System}

\subfile{subfiles/chapter_3/specifications/specifications.tex}

\chapter{BLOCK DIAGRAM OF THE SYSTEM}

\subfile{subfiles/chapter_4/block_diagram_intro/block_diagram_intro.tex}

\chapter{HARDWARE DESIGN}

Hardware design is carried out in four phases,

\begin{description}
    \item[1.]Mathematical model.
    \item[2.]Simulation and verification of algorithm.
    \item[3.]Calibration of sensor.
    \item[4.]Hardware Design. 
\end{description}

\section{Mathematical model}

The mathematical model serves the most important role throughout the project, as it is 
intended to solve the issues that persisted in previous research. 

This model is further divided into two sub-parts,

\subsection{Listener tracking model}

The listener tracking model is a combination of face detection and stereo vision 
technique for estimating depth.

\subsubsection{Face detection}

We must classify and sort out the entities from the rest of the objects from the 
surroundings to align the speakers properly.

In our case, these entities are people who are listening to the system. To classify 
them from other objects from surroundings, we implement the Harr cascade face detection 
algorithm to sort and cluster out these entities.

Harr cascade is a cascade classifier that implements a machine learning approach based 
on the Adaboost meta-algorithm.

The rectangular shape of the face is meaningful in initializing the classifier. Further, 
the algorithm focuses on the property that the eyes region is often darker than the face 
and nose region. The second feature proposes that the eyes are darker than the bridge of 
the nose. Similarly, this approach finds the entity's possible relations and features 
and records the features for further prediction. 

Once the face is detected, we can obtain the face's location from the origin (center of 
the image).

\subsubsection{Stereo Vision}

Stereo vision compares the information about a scene from two vantage points and 
examining relative positions of objects in the two panels.

An image can be termed as a grid of pixels within some range of indices. Using face 
detection, we narrowed down the object's position in the grid of pixels (x, y). 
Stereo vision gives two images of the same scene from different positions in the same 
plane. Each image gives the deviation of the image from the origin of that respective 
image.

\subsection{Angle and individual depth model}

\end{document}