\documentclass[../../../patent_v1.tex]{subfiles}

\begin{document}
    
\begin{description}
    \item[A.]\textbf{Surround sound systems}\\(United States Patents On, September 16, 2014)
\end{description}

This paper proposes an idea of the development of a system that comprises of receiver 
for receiving a multichannel spatial signal that comprises at least one surround channel.

This system comprises a directional ultrasonic transducer for emitting ultrasound 
towards the surface to reach a listening position via a reflection of the surface and a 
driver circuit to drive the ultrasonic transducer.

The proposed system is capable of producing virtual surround sound without requiring a 
speaker to be located.

\begin{description}
    \item[B.]\textbf{Shadow Sound System Embodied with Directional Ultrasonic Speaker}\\(ICISA.2013 on 2013) 
\end{description}

The paper talks about the usage of the ultrasonic speaker and computer vision system 
installed on a motorized mount that can freely change the speaker’s directions and 
altitude for a specific registered user.

The resulting system is proven to be able to track the registered 
user for providing user-selected sound contents without being 
interfered with by other people.

This method seems promising, but it requires individual hardware 
for each speaker, and the solution does not cover the implementation 
on multi-channel audio system efficiently.

\begin{description}
    \item[C.]\textbf{An Efficient Implementation of Acoustic Crosstalk Cancellation for 3D Audio Rendering}\\(IEEE China SIP on July 2014)    
\end{description}

In this paper given method the use of ultrasonic speaker and computer 
vision system installed on a motorized mount that can freely change 
the speaker’s directions and altitude for a specific registered user.

The resulting system is proven to be able to track the registered user 
for providing user-selected sound contents without being interfered 
by other people.

This method seems promising, but it requires individual hardware for 
each speaker and the solution does not cover the implementation on 
the multi-channel audio system efficiently.

\begin{description}
    \item[D.]\textbf{Multi-rate adaptive filtering for immersive audio}\\(IEEE Xplore  on February 2001)
\end{description}

This paper describes a method for implementing immersive audio 
rendering filters for single or multiple listeners and loudspeakers.

In particular, the paper focuses on the case of a single or two 
listeners with different loudspeaker arrays to determine the weighting 
vectors for the necessary FIR and IIR filters using the LMS (least-mean-squares) 
adaptive inverse algorithm.

It describes the transform-domain LMS adaptive inverse algorithm that is 
designed for crosstalk cancellation necessary in loudspeaker-based 
immersive audio rendering. 

The algorithm used in this paper is only for a single listener and for
only two loudspeakers.

\end{document}