\documentclass[../block_diagram_intro/block_diagram_intro.tex]{subfiles}

\begin{document}

Usually, a surround sound system contains two or more speakers to generate a sound 
effect of moving objects from one place to another.

Even if we direct the speakers towards the listener's direction, it is necessary to 
adjust each speaker's sound levels according to the depth of the listener from each 
speaker.

The audio processor unit assembles with 4 class AB audio amplifiers driven by two 
two-channel digital potentiometers for controlling each speaker's sound levels to adjust 
the sound pocket over the listener's head (ears).

\subsection{Audio amplifier}

An audio amplifier is a circuitry designed to increase the applied signal's magnitude 
to power a low resistance load (speakers).

Sound signals are applied to the non-inverting terminal of an amplifier through a 
voltage divider circuitry (potentiometer). This voltage divider adjusts the input 
signal's voltage levels, resulting in a change in volume levels at the output. This 
change is inversely proportional to the resistance at the wiper terminal of the voltage 
divider.

For this application, we are using LM386N-1 as our amplifier.

\subsection{Digital potentiometer}

Digital potentiometers mimic the analog functions of a mechanical potentiometer where 
micro-controllers or microprocessors control the resistance.

As we discussed, to adjust the audio amplifier's sound output, we adjust the input 
voltage given to the non-inverting terminal of the amplifier. Hence, we supply the 
audio signal to the amplifier through a digital potentiometer to increase and decrease 
input voltage and, hence, the speaker digitally using a micro-controller or a 
microprocessor.

For this application, we are using SPI-compatible MCP42010 Digital POT.

\end{document}