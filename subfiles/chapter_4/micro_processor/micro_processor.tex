\documentclass[../block_diagram_intro/block_diagram_intro.tex]{subfiles}

\begin{document}

Micro-processors acts as middleware between DAC and the hardware to control speakers.
In this project we are using raspberry pi 4B (Quad core Cortex-A72 64-bit @ 1.5 GHz clock) 
as our micro-processor. Webcams are connected through USB 2.0 of RPi.

First we find depth, deviation and height of the listener's face from the reference point
using opencv's frontal face harrcascade classifier followed by open source AA symmetry 
algorithm of geometry, where opencv also helps to classify between listener and other objects. 

Above process provides us three real-time variables,

\begin{description}
    \item[1.]Depth of the listener's face from the reference point.
    \item[2.]Deviation of the listener's face from center of the axis.
    \item[3.]Height of the center point of listener's face from the origin (reference point).  
\end{description}

Further using this real-time variables, and some constants (room dimensions and speaker positionings),
using customly designed geometrical algorithm we can calculate, panning and tilting angles, depth of 
the listener from each speaker.

Using panning and tilting angles we can rotate the servos to the required angles to direct the sound field
towards the listener. And using depth we can adjust the sound levels of the speakers by controlling
input voltage of amplifiers digitally.

\end{document}