\documentclass[../../../patent_v1.tex]{subfiles}

\begin{document}

\subsection{Depth estimation unit}

\subsubsection{Web Cam}

(LAPCARE LAPCAM)

\begin{figure}[ht]
    \centering
    \includegraphics[width=4cm]{web_cam.jpg}
    \caption{Web cam}
\end{figure}

\FloatBarrier

\begin{description}
    \item[1.]1280 x 720 pixels @ 720p resolution
    \item[2.]Automatic low light correction
    \item[3.]Plug and play linux compatible, High-Speed USB 2.0   
\end{description}


\subsection{Microprocessor}

Microprocessor serves an important role in DAC application, data processing estimation 
and controlling the response hardware in real time.

\begin{figure}[ht]
    \includegraphics[width=\columnwidth]{raspberry_pi.png}
    \caption{Microprocessor (Raspberry Pi 4B 4GB)}
\end{figure}

\FloatBarrier

Raspberry Pi 4B 4GB RAM model comes packed with, 

\begin{description}
    \item[1.]Quad core Cortex-A72 64-bit @ 1.5 GHz clock and uses ARM v8 architecture, 
    with 4GB LPDDR4-3200 SDRAM.
    \item[2.]2.4 and 5 GHz IEEE 802.11ac wireless wifi hardware.
    \item[3.]2 Micro HDMI ports.
    \item[4.]H.265 (4kp60 decode), H264 (1080p60 decode, 1080p30 encode).
    \item[5.]OpenGL ES 3.0 graphics.
    \item[6.]Micro-SD card slot for loading operating system and data storage.
    \item[7.]4 USB ports.
    \item[8.]Software PWM on all pins and Hardware on GPIO12, GPIO13, GPIO18, GPIO19.
    \item[9.]SPI
    \begin{description}
        \item[-]SPI0 : MOSI (GPIO10), MISO (GPIO09), SCLK (GPIO11), CE0 (GPIO08), CE1 (GPIO07)
        \item[-]SPI1 : MOSI (GPIO20), MISO (GPIO19), SCLK (GPIO21), CE0 (GPIO18), CE1 (GPIO17), CE2 (GPIO16).  
    \end{description}  
\end{description}

\subsection{Mechanical unit}

\subsubsection{Servo Motors}

(SG90 Servo)

\begin{figure}[ht]
    \centering
    \includegraphics[width=5cm]{servo.png}
    \caption{SG90 Servo}
\end{figure}

\FloatBarrier

\begin{description}
    \item[1.]180\si{\degree} rotation (90 in each direction).
    \item[2.]Torque 2.5 kg-cm
    \item[3.]Voltage 4.8-6 V  
    \item[4.]Speed 0.12 sec/60\si{\degree}
\end{description}

\subsection{Audio processor unit}

\subsubsection{Audio Amplifier}

(LM386N-1)

\begin{figure}[ht]
    \centering
    \includegraphics[width=5cm]{lm386.png}
    \caption{LM386N-1 pin-out}
\end{figure}

\FloatBarrier

\begin{description}
    \item[1.]Operating Supply Voltage (Vs) 4 - 12 V
    \item[2.]Voltage gain 20 - 200
    \item[3.]Output power 325 mW
\end{description}

\subsubsection{Digital Potentiometer}

(MCP42010)

\begin{figure}[ht]
    \centering
    \includegraphics[width=5cm]{mcp42010.png}
    \caption{MCP42010 pin-out}
\end{figure}

\FloatBarrier

\begin{description}
    \item[1.]Potentiometer values 10 k\si{\ohm}
    \item[2.]256 taps for each potentiometer
    \item[3.]2 channel
    \item[4.]SPI serial interface (mode 0, 0 and 1, 1)
    \item[5.]Single power operation (2.7V - 5.5V)
    \item[6.]Industrial temperature range: -40\si{\celsius} to +85\si{\celsius} 
    \item[7.]External temperature range: -40\si{\celsius} to +125\si{\celsius}    
\end{description}
 
\subsection{Speakers}

The final component is the speakers, which are mounted on servos and dynamically 
controlled sound using Audio Processor. They have mounted on four corners of the 
room to form a four-channel audio system. 

For this application, we are using 4\si{\ohm} speakers to deliver four-channel output.

\end{document}